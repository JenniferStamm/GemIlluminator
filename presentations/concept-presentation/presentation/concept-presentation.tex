\documentclass[aspectratio=169,10pt,t,german,xcolor=table]{beamer}
\usepackage{../../common/beamer-cgs-lecture}

\title{Gem Illuminator}
\author{Pascal Lange, Sebastian Koall, Jennifer Stamm}
\institute{\translate{Hasso Plattner Institute}}
\date{WiSe~2014/2015}

\subtitle{Konzeptpräsentation}
\titlegraphic{\includegraphics[width=\linewidth]{images/teaser}}

\begin{document}

\slidetitle
%\slidetableofcontents

%\slidesectionwithgraphic[bottom]{Cool Section Slide}{images/old_man}

\slideonetoone
{Initial Inspiration}
{
	\begin{figure}
		\centering
		\includegraphics[width=\textwidth, height=0.7\textheight, keepaspectratio]{images/mammut_cave}
		\caption{Game Programming 2013/14: 
Mammut \linebreak A highspeed gravity racer}
	\end{figure}
}
{
	\begin{itemize}
		\item Spielerüberforderung ...
		 \begin{itemize}
		 	\item durch hohe Geschwindigkeit
		 	\item durch Perspektivenwechsel (Drehung um 90 Grad)
		 	\item durch Gravitationsänderung
		 \end{itemize}
		 \item Cartoonähnliche Grafik 
		 \item Wenige Farbtöne
	\end{itemize}
}

\slideonetoone
{Initial Inspiration}
{
	\begin{figure}
		\centering
		\includegraphics[width=\textwidth, height=0.7\textheight, keepaspectratio]{images/krautscape_ks1}
		\caption{Krautscape \linebreak Multiplayer racing with procedural tracks}
	\end{figure}
}
{
	\begin{itemize}
		\item Prozedurale Streckengenerierung
		\item Beeinflussung der Streckengenerierung durch den Spieler
		\item Hohe Geschwindigkeit
		\item Cartoonähnliche Grafik
	\end{itemize}
}

\begin{frame}{Idea Development}
	\begin{itemize}
		\item 
	\end{itemize}
\end{frame}

\begin{frame}{Must Have}
\end{frame}

\begin{frame}{Should Have}
\end{frame}

\begin{frame}{Nice to Have}
\end{frame}

%\begin{frame}[allowframebreaks]{Bibliographie}
  \bibliographystyle{apalike}
  \bibliography{../../references}
  \vfill
\end{frame}

\end{document}

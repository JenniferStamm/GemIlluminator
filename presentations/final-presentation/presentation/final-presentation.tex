\documentclass[aspectratio=169,10pt,t,german,xcolor=table]{beamer}
\usepackage{../../common/beamer-cgs-lecture}

\title{Gem Illuminator}
\author{Pascal Lange, Sebastian Koall, Jennifer Stamm}
\institute{\translate{Hasso Plattner Institute}}
\date{WiSe~2014/2015}

\subtitle{Game Programming}
\titlegraphic{\includegraphics[width=\linewidth]{images/teaser}}

\begin{document}

% < 1m
\slidetitle
\section*{Abschlusspräsentation}

% 1m
\slideonetoonegraphics
{Motivation und Spielidee}
{images/start-situation}
{}
{images/lichteffekt-diffus-reflektion}
{}



% 3m
\begin{frame}{Demo}
	%TODO Aktueller Screenshot/Video
	\centering
	\includemedia[
	width=0.8\textwidth,height=0.8\textheight,
	activate=pageopen,
	flashvars={
		modestbranding=1 % no YT logo in control bar
		&autohide=1 % controlbar autohide
		&showinfo=0 % no title and other info before start
		&rel=0 % no related videos after end
	},
	url % Flash loaded from URL
	]{}{http://www.youtube.com/v/pz10FPVzF1k}
\end{frame}

% 1m
\begin{frame}{Architektur -- Zwischenstandspräsentation}
	\begin{figure}
		\centering
		\includegraphics[width=\textwidth, height=\textheight, keepaspectratio]{images/klassendiagramm}
	\end{figure}
\end{frame}

% 4m
\begin{frame}{Architektur -- Abschlusspräsentation}
	\begin{figure}
		\centering
		\includegraphics[width=\textwidth, height=\textheight, keepaspectratio]{images/klassendiagramm-final}
	\end{figure}
\end{frame}

% 2m
\begin{frame}{Game Features}
	%Must-Haves, Should-Haves, Nice-To-Haves
	\begin{itemize}
		\item Start-Menü
		\item Options-Menü
		\item Kristalllandschaft
		\item Wechselnde Lichtstrahlfarbe
		\item Einfluss auf die Kristallfarbe
		\item Reflektion bei Kollision
		\item Highscore
		\item Spielende bei Verlassen der Kristalllandschaft
	\end{itemize}
\end{frame}

% 2m
\begin{frame}{Computergraphic Features}

\end{frame}

% 2m
\begin{frame}{Herausfordungen}

\end{frame}

% 2m
\begin{frame}{Technische Einblicke 1}

\end{frame}

% 2m
\begin{frame}{Technische Einblicke 2}

\end{frame}

% 2m
\begin{frame}{Lessons learned}

\end{frame}

%\begin{frame}[allowframebreaks]{Bibliographie}
  \bibliographystyle{apalike}
  \bibliography{../../references}
  \vfill
\end{frame}

\end{document}
